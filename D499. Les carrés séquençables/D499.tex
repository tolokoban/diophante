\documentclass[10pt,a4paper,onecolumn]{article}
\usepackage[utf8]{inputenc}
\usepackage[francais]{babel}
\usepackage[margin=2cm]{geometry}
\usepackage[T1]{fontenc}
\usepackage{amsmath}
\usepackage{amsfonts}
\usepackage{amssymb}
\usepackage{listings}
\usepackage{graphicx}

\lstset{
  language=Java,
  showstringspaces=false,
  formfeed=newpage,
  frame=single,
  frameround=tttt
  tabsize=2,
  commentstyle=itshape,
  morekeywords={models, lambda, forms}
}

\author{Fabien PETITJEAN}
\title{D499. Les carrés séquençables}

\begin{document}
\maketitle
\begin{abstract}
{\it Ce problème, proposé par Michel Lafond, provient du site {\tt http://diophante.fr}.}

\vspace{.5cm}

Le carré $C_n$ de côté $n$ est dit séquençable si on peut le paver entièrement et sans chevauchement avec les rectangles $R_1,R_2, \ldots,R_k$  dont les dimensions $[a_1,a_2], [a_3,a_4],\ldots,[a_{2k-1} ,a_{2k}]$   sont à l’ordre près les entiers  $1,2,3,\ldots,2k$.
\begin{itemize}
\item Démontrer que le plus petit carré séquençable est $C_{11}$.
\item Trouver tous les entiers inférieurs ou égaux à 30 pour lesquels il existe un carré séquençable $C_n$.
\end{itemize}

Ci-après, à titre d'exemple, le carré séquençable $C_{13}$. Il est pavable avec les 5 rectangles $[1,2],[3,8],[4,5],[6,10]$ et $[7,9]$ dont les dimensions sont 1, 2, 3, 4, 5, 6, 7, 8, 9, 10.

\begin{tabular}{|c|c|c|c|c|c|c|c|c|c|c|c|c|}
\hline A&A&A&A&A&A&A&A&B&B&B&B&B \\
\hline A&A&A&A&A&A&A&A&B&B&B&B&B \\
\hline A&A&A&A&A&A&A&A&B&B&B&B&B \\
\hline C&C&C&C&C&C&D&D&B&B&B&B&B \\
\hline C&C&C&C&C&C&E&E&E&E&E&E&E \\
\hline C&C&C&C&C&C&E&E&E&E&E&E&E \\
\hline C&C&C&C&C&C&E&E&E&E&E&E&E \\
\hline C&C&C&C&C&C&E&E&E&E&E&E&E \\
\hline C&C&C&C&C&C&E&E&E&E&E&E&E \\
\hline C&C&C&C&C&C&E&E&E&E&E&E&E \\
\hline C&C&C&C&C&C&E&E&E&E&E&E&E \\
\hline C&C&C&C&C&C&E&E&E&E&E&E&E \\
\hline C&C&C&C&C&C&E&E&E&E&E&E&E \\
\hline
\end{tabular}
\end{abstract}

\section{Solution}
Recherchons d'abord le nombre minimal de rectangles nécessaires à paver un carré dans les conditions du problème.


\begin{enumerate}
\item Il est évident qu'un rectangle ne suffit pas.
\item Si on pouvait paver un carré avec deux rectangles, ces rectangles auraient un côté en commun, donc de même longueur, ce qui est contraire à l'énoncé.
\item Si on considère trois rectangles, on a les 5 configurations suivantes :\\
\begin{itemize}
\item $1 \times 2 + 3 \times 4 + 5 \times 6 = 44$.
\item $1 \times 3 + 2 \times 4 + 5 \times 6 = 41$.
\item $1 \times 4 + 2 \times 3 + 5 \times 6 = 40$.
\item $1 \times 5 + 2 \times 3 + 4 \times 6 = 35$.
\item $1 \times 6 + 2 \times 3 + 4 \times 5 = 32$.
\end{itemize}
Comme on le constate, la somme des aires des trois rectangles ne donne jamais un carré. Donc on ne peut paver aucun carré avec trois rectangles.
\item Avec le même raisonnement, on remarque que sur les 7 configurations possibles, 2 pourraient paver une carré de côté 10 et 9 respectivement :
\begin{itemize}
\item \fbox{$1 \times 2 + 3 \times 4 + 5 \times 6 + 7 \times 8 = 100 = 10^2$}.
\item $1 \times 3 + 2 \times 4 + 5 \times 6 + 7 \times 8 = 97$.
\item $1 \times 4 + 2 \times 3 + 5 \times 6 + 7 \times 8 = 96$.
\item $1 \times 5 + 2 \times 3 + 4 \times 6 + 7 \times 8 = 91$.
\item $1 \times 6 + 2 \times 3 + 4 \times 5 + 7 \times 8 = 88$.
\item \fbox{$1 \times 7 + 2 \times 3 + 4 \times 5 + 6 \times 8 = 81 = 9^2$}.
\item $1 \times 8 + 2 \times 3 + 4 \times 5 + 6 \times 7 = 76$.
\end{itemize}
Dans le premier cas, on a les rectangles $R(1,2)$, $R(3,4)$, $R(5,6)$ et R$(7,8)$. Aucun de ces rectangles n'est assez grand pour couvrir à lui seul un côté du carré (dont la longueur est 10). Comme c'est vrai pour chacun des 4 côtés, il nous faut donc au moins deux rectangles par côté. Et avec 4 rectangles, il est impossible d'en avoir plus de 2. En effet, si on utilise 3 rectangles pour un côté, on aura forcément un côté avec un seul rectangle, ce qui est impossible.

Dans ce cas, chaque rectangle doit se trouver dans un coin du carré. On peut donc considérer que le rectangle $R(1,2)$ est situé en haut à gauche du carré de $10\times 10$. Il mesure 1 en largeur et 2 en hauteur. A sa droite, il faudrait un rectangle avec un bord de longueur 9, mais il n'en existe aucun de cette taille. Cette configuration est donc impossible.

Dans le second cas, on a les rectangles $R(1,7)$, $R(2,3)$, $R(4,5)$ et R$(6,8)$. En utilisant le même raisonnement, on place le $R(1,7)$ en haut à gauche et il doit forcément avoir comme immédiat voisin de droite le $R(8,6)$. On peut aussi placer le $R(2,3)$ en bas à gauche. Mais alor, il nous faudrait un rectangle de largeur $6$ en bas à droite et malheureusement, on l'a déjà utilisé en haut à droite.

\item .0.
\end{enumerate}

%\lstinputlisting{perms.js}


\end{document}