\documentclass[10pt,a4paper]{article}
\usepackage[utf8]{inputenc}
\usepackage[francais]{babel}
\usepackage[T1]{fontenc}
\usepackage{amsmath}
\usepackage{amsfonts}
\usepackage{amssymb}
\author{Fabien PETITJEAN}
\title{Podcast}
\begin{document}
\noindent
\begin{description}
\item[Problème 1 ($\star$ $\star$) ]
Je suis un entier naturel de 6 chiffres. On supprime l’un de mes chiffres  et le nombre résultant qui ne commence pas par un zéro me divise. On continue le processus en supprimant un chiffre  à chaque étape et le nombre résultant qui ne commence jamais par un zéro divise toujours celui qui le précède.On s’arrête quand il reste un seul chiffre et  les quotients obtenus par les cinq divisions successives sont tous distincts. Qui suis-je ?
\end{description}

\begin{description}
\item[Problème 2 ($\star\star\star$)]
Je suis un entier naturel de 11 chiffres qui est un multiple de 4. On choisit un certain entier $k < 12$. On supprime mon chiffre $u$ des unités et on ajoute la quantité $ku$ au nombre amputé à 10 chiffres. On opère de la même manière avec le nombre résultant dont on ampute le chiffre des unités et auquel on ajoute le produit de ce chiffre par le même entier $k$ jusqu’au moment où l’on obtient un nombre premier qui se répète indéfiniment. On recommence ces amputations en série avec cinq autres valeurs de $k$ toutes distinctes et inférieures à 12.A chaque fois, on obtient un nombre premier qui se répète. Qui suis-je ?
\end{description}
\end{abstract}

\section{Problème 1}


\section{Problème 2}
Comme précédemment, il semble plus pratique de prendre le problème dans l'autre sens.
Nous allons donc commencer par rechercher les nombres premiers qui \og bouclent \fg.
Soit $p = 10n + u$ un tel nombre premier. On a donc l'égalité suivante : 
$$10n + u =  n + ku \text{ avec } \left\{\begin{array}{r}
u\in\{1, 3, 5, 7, 9\} \\
k\in\{2, 3, \ldots, 11\} \\
\end{array}
\right.$$
$u$ représentant le chiffre des unités d'un nombre premier, il ne peut donc pas être pair. Par ailleurs, $k$ ne peut être nul, ni égal à 1, sinon cela entrainerait que le nombre premier soit nul.

En écrivant cette égalité différemment, on obtient : $n = \frac{u(k - 1)}9$.
Cette division par 9 implique les contraintes suivantes :
\begin{itemize}
\item soit $u=9$ ,
\item soit $k=10$,
\item soit $u=3$ et $k=4$.
\end{itemize}

\begin{table}[!ht]
\caption{Les nombres premiers racines\label{tbl:roots}}
\center{
\begin{tabular}{|c|r|r|r|r|r|r|r|r|r|}
\hline $u$ & 1 & 3 & 9 & 3 & 9 & 9 & 9 & 9 & 9 \\
\hline $k$ & 10 & 4 & 2 & 7 & 3 & 6 & 8 & 9 & 11 \\
\hline $p$ & 11 & 13 & 19 & 23 & 29 & 59 & 79 & 89 & 109 \\
\hline
\end{tabular}
}
\end{table}

On en déduit les nombres premiers \emph{racines}, c'est-à-dire les seuls qui \og bouclent \fg. Ils sont répertoriés dans la table \ref{tbl:roots}.

Nous remarquons que les valeurs de $k$ sont toutes distinctes, ce qui est rassurant en regard de l'énoncé qui suggère l'existence de 6 valeurs distinctes de $k$ qui mènent vers des nombres premiers qui \og bouclent \fg.

Nous allons maintenant tenter de remonter depuis ces racines jusqu'à des nombres de 11 chiffres multiples de 4.

Représentons une telle remontée par la suite $(a_n)_{n\in\mathbb{N}}$ telle que $a_0$ est un des nombres premiers qui \og bouclent \fg. Notons $u_i$, le chiffre des unités du nombre $a_i$. Alors la règle de transformation des nombres nous donne :
$\forall n > 0, \: a_{n-1} = \frac{a_n - u_n}{10} + k u_n \iff 
a_n = 10a_{n-1} + (1 - 10k)u_{n-1}$. D'où :
$$
\begin{array}{ccl}
a_1 &=& 10a_0 + (1 - 10k)u_1 \\
a_2 &=& 10a_1 + (1 - 10k)u_2 \\
&\cdots & \\
a_n &=& 10^n a_0 + (1 - 10k) \sum_{i=1}^n 10^{n-i}u_i
\end{array}
$$
Nous obtenons donc une équation à partir de laquelle nous allons tenter de déduire des informations concernant le nombre à 11 chiffre recherché.

\begin{equation}\label{eq:suite}
  a_n = 10^n a_0 + (1 - 10k) \underbrace{[u_1 u_2 \ldots u_n]}_{
    \begin{array}{c}
    \text{nombre avec}\\
    \text{un maximum}\\
    \text{de $n$ chiffres}\\
    \end{array}
  }
\end{equation}

Puisque nous cherchons un nombre de 11 chiffres, nous devons satisfaire l'encadrement suivant : $10^{11} \leq a_n < 10^{12}$.

Prenons le cas où $p=a_0=11$, d'après la table \ref{tbl:roots} on a forcément $k=10$
et alors l'équation \ref{eq:suite} devient :
$$a_n = 11 \times 10^n - 99[u_1 u_2 \ldots u_n]$$

Posons $N = [u_1u_2 \ldots u_n]$. Si $u_1 > 0$ alors $N \geq 10^n$ et donc 
$a_n < 11 \times 10^n -99 \times 10^n = -88 \times 10^n < 0$. On vient donc de montrer par l'absurde que $u_1 = 0$ ce qui implique que 
$10^n > N \geq 10^{n-1} \iff -88 \times 10^n < a_n \leq 1.1 \times 10^n$.
Et puisque $a_n$ doit avoir 11 chiffres, on obtient finalement cet encadrement :
$$10^11 \leq a_n < 1.1 \times 10^n$$.
\appendix

\section{Annexes}
Ces annexes regroupent les programmes (écrits en Python3) utilisés ou pouvant être utilisés.

\subsection{Rechercher les nombres premiers racines} 
Il s'agit des nombres qui \og bouclent \fg.

La table \ref{tbl:roots} en page \pageref{tbl:roots} se calcule facilement à l'aide du programme suivant :

\lstinputlisting{pb2root.py}

\end{document}
